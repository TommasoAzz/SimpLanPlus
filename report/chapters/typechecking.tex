\documentclass[../report.tex]{subfiles}
\begin{document}

\chapter{Type checking}\label{c:typechecking}
\section{Correttezza dei tipi}\label{s:correttezza-tipi}
La correttezza dei tipi viene gestista tramite il metodo \verb|Node typeCheck()| che restituisce un nodo rappresentante il tipo del nodo (istanza di \verb|Node|) su cui si è invocato, null in caso quel nodo non debba avere associato un tipo o  un'eccezione in caso di errori durante il processo.\\
Il valore \verb|null| viene ritornato da tutti i nodi che hanno come supertipo \verb|TypeNode|, dagli argomenti delle funzioni, dalle dichiarazione di variabili e funzioni.\\
Le istruzioni, per scelta implementativa, ritornano il tipo \verb|void| ma ci sono alcune eccezioni:
\begin{itemize}
    \item \verb|RetNode| che ritorna il tipo dell'espressione, \verb|void| nel caso non ce ne sia alcuna;
    \item \verb|CallNode| che ritorna il tipo di ritorno dichiarato dalla funzione;
    \item \verb|IteNode| che ritorna il tipo comune ai due rami (che deve essere quindi uguale);
    \item \verb|BlockNode| che ritorna \verb|void| in assenza di istruzioni o quando non vi è n\'e \verb|return| n\'e \textit{if-then-else} altrimenti ritorna il tipo di \verb|RetNode| o di \verb|IteNode|. 
\end{itemize}
\subsection{Tipi}\label{s:tipi}
Come brevemente anticipato in \hyperref[s:dichiarazioni]{Sezione 2.2 Dichiarazioni}, i tipi implementati nel linguaggio sono:
\begin{itemize}
    \item \verb|bool|: rappresenta un tipo booleano \verb|true| o \verb|false|;
    \item \verb|int|: rappresenta un tipo intero;
    \item \verb|pointer|: rappresenta un tipo puntatore che punta a un tipo booleano, intero o a un altro tipo puntatore;
    \item \verb|void|: rappresenta il tipo vuoto;
    \item \verb|function|: tipo utilizzato per le funzioni, contiene la lista dei tipi dei parametri e il tipo ritornato.
\end{itemize}
\end{document}

