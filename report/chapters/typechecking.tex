\documentclass[../main.tex]{subfiles}
\begin{document}

\chapter{Type checking}\label{c:typechecking}
\section{Correttezza dei tipi}
La correttezza dei tipi viene gestista tramite il metodo \verb|Node typeCheck()| che restituisce un nodo rappresentante il tipo del nodo su cui si \`e invocata, null in caso quel nodo non debba avere associato un tipo o  un'eccezione in caso di errori durante il processo.
\subsection{Tipi}
I tipi implementati nel linguaggio sono:
\begin{itemize}
    \item \textbf{Boolean}: rappresenta un tipo booleano \verb|true| o \verb|false|;
    \item \textbf{Integer}: rappresenta un tipo intero;
    \item \textbf{Pointer}: rappresenta un tipo puntatore che punta a un tipo booleano, intero o a un altro tipo puntatore;
    \item \textbf{Function}: tipo utilizzato per le funzioni, contiene la lista dei tipi dei parametri, il tipo ritornato e la lista degli effeti che i parametri subiscono all'invocazione della funzione.
\end{itemize}
\end{document}

