\documentclass[../report.tex]{subfiles}
\begin{document}

\chapter{Esempi}\label{c:esempi}
Nelle seguenti sezioni sono riportati alcuni dei 31 esempi che sono stati utilizzati per accertarsi che il compilatore e l'interprete del linguaggio \textbf{SimpLanPlus} rispettasse le specifiche richieste.
\section{Esempio 1}\label{s:esempio1}
\begin{lstlisting}
{
    ^^int x;
    ^int y = new int;
    y^ = 1;
    x = new ^int;
    x^ = y;
    print x^^;
}
\end{lstlisting}
L'esempio presente nel file \verb|examples/example1.simplan| mostra il corretto uso dei puntatori.
Vengono creati due puntatori, \verb|x| che punta a sua volta a un puntatore di interi e \verb|y| che punta a un intero.
Le aree di memoria puntate dai puntatori vengono scritte e ciò viene verificato stampando il valore intero puntato dal puntatore puntato da \verb|x|.
Infine, nell'istruzione \verb|x^ = y| oltre a fare la copia del valore della variabile puntatore \verb|y| nella memoria puntata da \verb|x^| vengono copiati anche i relativi effetti.

\section{Esempio 2}\label{s:esempio2}
\begin{lstlisting}
{
    ^int x = new int;
    x^ = 1;
    delete x;
    y = x^; // this is wrong!
}
\end{lstlisting}
L'esempio presente nel file \verb|examples/example2.simplan| mostra il riconoscimento dell'utilizzo di un identificatore mai dichiarato.
L'assegnazione del valore \verb|x^| alla variabile \verb|y| mai dichiarata viene riconosciuta come errore semantico e correttamente riportata con un avviso \verb|Missing declaration for ID: y|.
Inoltre, tramite l'analisi degli effetti, il compilatore riconosce che la variabile \verb|x| viene usata dopo essere stata cancellata mostrando all'utente il messaggio \verb|Variable x is used after deletion|.

\section{Esempio 3}\label{s:esempio3}
\begin{lstlisting}
{
    ^int x = new int;
    void f(^int x, ^int y) {
        delete x;
        delete y;
    }

    f(x,x);
}
\end{lstlisting}
L'esempio presente nel file \verb|examples/example3.simplan| mostra il riconoscimento del problema dell'aliasing.
La funzione \verb|void f(^int x, ^int y)| prende in input due puntatori e li elimina.
Quando la funzione viene invocata passandole lo stesso puntatore due volte \verb|f(x, x)| il compilatore riconosce, correttamente, che il puntatore \verb|x| verrebbe eliminato due volte e lo comunica all'utente tramite il messaggio \verb|The dereferenced pointer x^ is used after deletion|.

\section{Esempio 8}\label{s:esempio8}
\begin{lstlisting}
{
    void f(int m, int n) {
        if (m>n) {
            print(m+n);
        }
        else {
            int x = 1;
            f(m+1,n+1);
        }
    }
    f(5,4);
}
\end{lstlisting}
L'esempio presente nel file \verb|examples/example8.simplan| mostra il corretto funzionamento di una chiamata ricorsiva.
In caso si invertissero i parametri della chiamata, passando da \verb|f(5, 4)| a \verb|f(4, 5)|, questa funzione, correttamente, non potrebbe terminare.
Essendo la memoria dell'interprete finita, prima o poi, verrebbe mostrato il messaggio di errore \verb|Error: Reached max memory limit|.

\section{Esempio 16}\label{s:esempio16}
\begin{lstlisting}
{
    void f() {
        print 1;
        g();
    }
    void g() {
        print 2;
        f();
    }

    f();
}
\end{lstlisting}
L'esempio presente nel file \verb|examples/example16.simplan| dimostra che la mutua ricorsione non è permessa in \textbf{SimpLanPlus}.
Infatti, lanciando questo programma viene visualizzato il messaggio di errore \verb|Missing declaration for ID: g|.

\section{Esempio 20}\label{s:esempio20}
\begin{lstlisting}
{
    int fib(int n) {
        if (n <= 1) {
            return n;
        }
        return fib(n-1) + fib(n-2);
    }

    print fib(10);
}
\end{lstlisting}
L'esempio presente nel file \verb|examples/example20.simplan| non ha nulla di speciale in sé, ma è un esempio di codice \textbf{SimpLanPlus} meno didattico e scritto con il solo scopo di verificare che funzionino il compilatore e l'interprete.
Come si può dedurre dal nome della funzione infatti, il seguente codice implementa la versione ricorsiva ``dummy" della funzione per il calcolo dell'$n$-esimo numero della serie di Fibonacci.
Nonostante tutto, è interessante il caso base di questa funzione (\verb|if (n <= 1) { return n; }|), perché ha portato a riconoscere la necessità di introdurre il registro \textbf{\$bsp} descritto in \hyperref[ss:registri]{Sezione 6.2.2 Registri}.
Infatti, sebbene l'istruzione \verb|return| possa semplicemente memorizzare il valore ritornato in \textbf{\$a0}, deve anche saltare tutte le istruzioni che vengono dopo, in questo caso le ulteriori chiamate ricorsive presenti dopo il costrutto \verb|if|.

\section{Esempio 25}\label{s:esempio25}
\begin{lstlisting}
{
    ^int a = new int;
    ^int b = new int;
    void f(^int x, ^int y, int n) {
        if (n == 0) {
            delete y;
        }
        else {
            ^int w = new int;
            f(w, x, n-1);
            delete y;
        }
    }

    f(a, b, 1);
}
\end{lstlisting}
L'esempio presente nel file \verb|examples/example25.simplan| infine, mostra come sia necessario avere il metodo di punto fisso (MPF) per calcolare gli effetti applicati ai parametri formali delle funzioni.
Senza il MPF, \verb|f| avrebbe avuto come effetti \verb|x| $\rightmapsto{} rw$, \verb|y| $\rightmapsto{} d$ e \verb|n| $\rightmapsto{} rw$, non riconoscendo che la chiamata ricorsiva nel ramo \verb|else| porta anche \verb|x| ad essere cancellato.
Con l'implementazione corretta che è stata fatta, tentando di accedere in lettura alle variabili \verb|a| e \verb|b| dopo il codice mostrato in esempio viene segnalato un errore perché le variabili potrebbero essere state cancellate.
\end{document}

