\documentclass[../report.tex]{subfiles}
\begin{document}

\chapter{Introduzione}\label{c:introduzione}
\section{Scopo del progetto}\label{s:scopo-progetto}
Lo scopo del progetto è stato quello di realizzare un compilatore e un interprete per un linguaggio imperativo chiamato \textbf{SimpLanPlus}.\\
\noindent
L'interprete è chiamato \textbf{SimpLanPlus Virtual Machine (SVM)} ed esegue un linguaggio assomigliante, per molti aspetti, ad Assembly, appositamente creato per il suo funzionamento, che da qui in avanti chiameremo \textbf{SVM-Assembly}.

\subsection{Il linguaggio}\label{ss:linguaggio}
\textbf{SimpLanPlus} è un linguaggio:
\begin{itemize}
    \item staticamente tipato;
    \item non ad oggetti;
    \item con gestione della memoria manuale (come, ad esempio, in C/C++);
    \item con la possibilità di definire funzioni, anche ricorsive, ma non mutualmente ricorsive.
\end{itemize}

\subsection{Il compilatore}\label{ss:compilatore}
Il compilatore di \textbf{SimpLanPlus} svolge i seguenti step (in ordine) prima di generare il codice \textbf{SVM-Assembly} corrispondente:
\begin{enumerate}
    \item analisi lessicale, in cui il codice sorgente viene trasformato in \textit{token}, ovvero piccoli blocchi logici, che contribuiscono alla creazione dell'\textbf{Abstract Syntax Tree (AST)};
    \item analisi semantica, in cui viene verificata la correttezza semantica del programma in input (ad esempio, che le variabili vengano utilizzate solo dopo una loro dichiarazione);
    \item \textit{type checking}, in cui viene controllato che i tipi all'interno delle espressioni, nelle assegnazioni e nelle istruzioni siano concordi a quanto atteso;
    \item analisi degli effetti, in cui viene verificato lo stato delle variabili e dei puntatori nella memoria (inizializzato $\bot$, letto/scritto $rw$, cancellato $d$, errore $\top$).
\end{enumerate}

\subsection{L'interprete}\label{ss:interprete}
L'interprete di \textbf{SimpLanPlus}, \textbf{SVM}, prende in input il codice \textbf{SVM-Assembly} generato dal compilatore e illustrato nella precedente sezione, eseguendolo un'istruzione alla volta.
Il programma viene eseguito in una ``memoria virtuale'' (simulata dal codice che implementa l'interprete) la cui dimensione è configurabile (come farlo è illustrato nel \hyperref[c:simplanplus-java]{Capitolo 7 SimpLanPlus.java}).

\section{Tecnologie utilizzate}\label{s:teconologie-utilizzate}
Le tecnologie principali utilizzate per il progetto sono le seguenti:
\begin{itemize}
    \item \textbf{Java 11}, con cui sono scritti gli interi compilatore e interprete;
    \item \textbf{ANTLR 4.9.2}, per implementare in maniera automatica la parte di analisi lessicale e il parser, sia per \textbf{SimpLanPlus}, che per il linguaggio \textbf{SVM-Assembly};
    \item \textbf{Maven (wrapper) 3.6.3}, per gestire le dipendenze, il processo di build e di test di tutte le parti del progetto.
\end{itemize}

\end{document}
