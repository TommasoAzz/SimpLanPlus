\documentclass[../report.tex]{subfiles}
\begin{document}

\chapter{Introduzione}\label{c:introduzione}
\section{Scopo del progetto}\label{s:scopo-progetto}
Lo scopo del progetto è stato quello di realizzare un compilatore e un interprete per un linguaggio imperativo chiamato \textbf{SimpLanPlus}.

\subsection{Il linguaggio}
\textbf{SimpLanPlus} è un linguaggio:
\begin{itemize}
    \item staticamente tipato;
    \item non ad oggetti;
    \item con gestione della memoria manuale (come in C/C++);
    \item con la possibilità di definire funzioni, anche ricorsive, ma non mutualmente ricorsive.
\end{itemize}

\subsection{Il compilatore}
Il compilatore di \textbf{SimpLanPlus} svolge i seguenti step (in ordine) prima di generare il codice simil-assembly corrispondente:
\begin{enumerate}
    \item analisi lessicale, in cui il codice sorgente viene trasformato in \textit{token}, ovvero piccoli blocchi logici, che servono poi a contribuire a creare l'\textbf{Abstract Syntax Tree (AST)};
    \item analisi semantica, in cui viene verificata la correttezza semantica del programma in input (ad esempio, che le variabili vengano utilizzate solo dopo una loro dichiarazione);
    \item analisi degli effetti, in cui viene verificato lo stato delle variabili e dei puntatori sullo stack (inizializzato $\bot$, letto/scritto $rw$, cancellato $d$, errore $\top$);
    \item \textit{type checking}, in cui viene controllato che i tipi all'interno delle espressioni, nelle assegnazioni e nelle istruzioni siano concordi a quanto atteso.
\end{enumerate}

\subsection{L'interprete}

\section{Tecnologie utilizzate}\label{s:teconologie-utilizzate}

\end{document}
