\documentclass[../report.tex]{subfiles}
\begin{document}

\begin{abstract}
    \noindent
    SimpLanPlus \`e un linguaggio ideato per il corso di Compilatori e Interpreti della LM Informatica A.A. 2020/21. Il linguaggio adotta il paradigma imperativo, \`e tipato, permette la definizioni di funzioni anche ricorsive, l'utilizzo di puntatori e la gestione della memoria \`e manuale. Per questo progetto \`e stata descritta la grammatica del linguaggio e sono state implementate tecniche e metodologie per eseguire l'analisi semantica, con particolare attenzione all'analisi degli effetti e al controllo dei tipi. \`E stato poi definito un linguaggio intermedio simil-assembly che viene preso in input  da un interprete anch'esso progettato e realizzato all'interno di questo progetto. La progettazione e la realizzazione dell'interprete hanno richiesto lo studio del funzionamento di una macchina a pila nella quale \`e permesso l'utilizzo di registri.
\end{abstract}

\end{document}

