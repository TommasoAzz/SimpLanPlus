\documentclass[../report.tex]{subfiles}
\begin{document}

\chapter{Analisi semantica}\label{c:analisi-semantica}
\section{Struttura ambiente}\label{s:struttura-ambiente}
L'ambiente è definito all'interno del nostro compilatore attraverso le classi \verb|Environment| e \verb|STEntry|, la prima che rappresenta l'ambiente nel suo intero, la seconda invece una singola \textit{entry}.
Per semplicità, \verb|Environment| corrisponde sia al c.d. ambiente $\Gamma{}$, utilizzato per il type checking, che all'ambiente $\Sigma{}$ utilizzato per l'analisi degli effetti, di cui viene trattato più approfonditamente in \hyperref[c:analisi-effetti]{Capitolo 4 Analisi degli effetti}. In questo capitolo verrà trattato solo ciò che concerne l'analisi semantica in senso stretto.

\subsection[Classe Environment]{Classe \texttt{Environment}}
\verb|Environment| è una classe molto ricca di funzionalità che permette di gestire ad alto livello la \textit{Symbol Table}. Tale \textit{Symbol Table} è implementata come uno \textit{stack} di \textit{hash map}, in cui ogni elemento dello \textit{stack} (implementato come una \textit{array} dinamico) rappresenta uno scope e ogni elemento di una \textit{hash map} associa la stringa di un identificatore con la corrispondente \textit{entry}, rappresentata da un'istanza di \verb|STEntry|.
Ci sono inoltre altri campi ausiliari che memorizzano il livello di annidamento corrente e l'\textit{offset} nello \textit{scope} corrente per la generazione del codice \textbf{SVM-Assembly}.\\
I metodi per la gestione della \textit{Symbol Table} sono:
\begin{itemize}
    \item \verb|void pushNewScope()| che corrisponde all'operazione $\Gamma{}\cdot{}[\;]$, la quale incrementa contestualmente il livello di annidamento;
    \item \verb|STEntry addNewDeclaration(final String id, final TypeNode type)| che corrisponde all'operazione $\Gamma{}[$\verb|id| $ \rightmapsto{}$\verb|type|$]$, aggiornando contestualmente l'offset in caso $type$ non corrisponda ad un tipo funzione;
    \item \verb|STEntry lookup(final String id)| che corrisponde all'operazione $\Gamma($\verb|id|$)$ che restituisce la prima \verb|STEntry| corrispondente all'identificatore $id$ a partire dal top dello stack.
    \item \verb|void popScope()| che corrisponde all'operazione $\Gamma \leftarrow{} \Gamma'$ partendo da $\Gamma = \Gamma'\cdot{}top(\Gamma)$.
\end{itemize}

\subsection[Classe STEntry]{Classe \texttt{STEntry}}
Come specificato nella precedente sezione, \verb|STEntry| raccoglie i dati relativi ad una singola entry della \textit{Symbol Table}, sia per $\Gamma{}$ che per $\Sigma{}$.
I campi dati relativi alla gestione delle entry della \textit{Symbol Table} sono:
\begin{itemize}
    \item \verb|nestingLevel| che rappresenta il livello di annidamento della variabile o funzione all'interno del codice sorgente;
    \item \verb|type| che rappresenta il tipo correlato all'identificatore;
    \item \verb|offset| che è la posizione nel \textit{frame} rispetto all'Access Link.
\end{itemize}

\section{Controllo errori semantici}
Il controllo degli errori semantici e la costruzione della \textit{Symbol Table} viene fatta tramite il metodo \verb|ArrayList<SemanticError> checkSemantics(Environment env)| presente in ogni nodo dell'AST.
L'istanza di \verb|Environment| viene creata e passata alla radice dell'AST all'interno della classe \verb|SimpLanPlus|, illustrata più in dettaglio nel \hyperref[c:simplanplus-java]{Capitolo 7 SimpLanPlus.java}.
Ogni nodo si occupa di verificare la correttezza di sé e, se dispone di nodi figli nell'AST, lo invoca anche su di loro.
In caso di presenza di errori, essi vengono segnalati con la creazione di un'istanza di \verb|SemanticError|, una classe di utilità che contiene il messaggio d'errore da segnalare all'utente.
Tutte le istanze di \verb|SemanticError| vengono collezionate e ritornate alla fine dell'esecuzione del metodo \verb|checkSemantics| sul nodo corrente.
Conseguentemente, per quanto spiegato finora, ogni nodo ritorna gli errori di sé e dei suoi nodi figli.\\
\noindent
L'implementazione dei metodi checkSemantics ricalca quanto visto durante le lezioni sull'analisi semantica del corso di ``Compilatori e interpreti", i cui riferimenti alle slide si trovano in fondo al presente documento.


\end{document}

