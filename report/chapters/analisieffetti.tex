\documentclass[../report.tex]{subfiles}
\begin{document}

\chapter{Analisi degli effetti}\label{c:analisi-effetti}
\section{Struttura ambiente}\label{s:struttura-ambiente-effetti}
Essendo l'ambiente dell'analisi degli effetti correlato dal punto di vista implementativo con quello dell'analisi semantica, vale quanto descritto in \hyperref[s:struttura-ambiente]{Sezione 3.1 Struttura ambiente}.

\subsection[Classe Effect]{Classe \texttt{Effect}}\label{ss:effect-effetti}
La classe \verb|Effect| rappresenta un effetto applicato ad una variabile.
Contiene un solo vero e proprio campo dati, \verb|value| (di tipo intero, in modo che esista un ordinamento), che memorizza l'effetto corrente.
Ci sono altri campi dati, statici, che sono istanze della stessa classe e rappresentano gli effetti possibili:
\begin{itemize}
    \item \verb|INITIALIZED| ($\bot$, con \verb|value| $= 0$), assegnato ad una variabile quando viene dichiarata ma non allocata;
    \item \verb|READ_WRITE| ($rw$, con \verb|value| $= 1$), assegnato ad una variabile quando viene letta o scritta;
    \item \verb|DELETE| ($d$, con \verb|value| $= 2$), assegnata ad una variabile di tipo puntatore quando viene cancellata la memoria puntata;
    \item \verb|ERROR| ($\top$, con \verb|value| $= 3$), assegnata ad una variabile il cui stato è inconsistente e irreparabile.
\end{itemize}
I metodi disponibili per operare su istanze di tipo \verb|Effect| sono tutti statici e sono i seguenti:
\begin{itemize}
    \item \verb|Effect max(Effect e1, Effect e2)| che ritorna il massimo fra \verb|e1| e \verb|e2| (vengono confrontati i campi \verb|value|)
    \item \verb|Effect seq(Effect e1, Effect e2)| che ritorna \verb|max(e1, e2)| se $\leq{} rw$, $d$ se \verb|e1| $\leq{} rw$ e \verb|e2| $= d$ oppure \verb|e1| $= d$ e \verb|e2| $= \bot$, $\top$ in tutti gli altri casi;
    \item \verb|Effect par(Effect e1, Effect e2)| che ritorna il massimo fra gli effetti ritornati dalle invocazioni \verb|seq(e1, e2)| e \verb|seq(e2, e1)|.
\end{itemize}

\subsection[Classe STEntry]{Classe \texttt{STEntry}}\label{ss:stentry-effetti}
Oltre ai campi dati illustrati in \hyperref[ss:stentry]{Sezione 3.1.2 \texttt{STEntry}}, sono stati aggiunti due campi dati \verb|variableStatus| e \verb|functionStatus| (rispettivamente di tipo \verb|List<Effect>| e \verb|List<List<Effect>>|) rappresentanti il primo gli effetti correnti sulle aree di memoria occupate dalla variabile e il secondo gli effetti applicati alle aree di memoria degli argomenti della funzione, cui fa riferimento la \textit{entry}.
In caso la \textit{entry} si riferisca ad una variabile, \verb|functionStatus| è una lista vuota e non viene in alcun modo utilizzata; viceversa nel caso sia una funzione.\\
La scelta di utilizzare una \verb|List<Effect>| è stata resa necessaria per poter gestire allo stesso modo variabili non puntatori (che avranno a loro associate \verb|variableStatus| di lunghezza 1) e variabili puntatori (che avranno \verb|variableStatus| di lunghezza pari al massimo numero di dereferenziazioni possibili su di esse, certamente $> 1$).


\subsection[Classe Environment]{Classe \texttt{Environment}}\label{ss:environment-effetti}
In \verb|Environment|, oltre a quanto descritto in \hyperref[ss:environment]{Sezione 3.1.1 Classe \texttt{Environment}}, ci sono ulteriori metodi che si occupano della gestione dell'analisi degli effetti.\\
\noindent
I principali sono descritti di seguito:
\begin{itemize}
    \item \verb|Environment max(Environment env1, Environment env2)| che per ogni scope, a partire da quello di livello $0$, per ogni variabile presente sia in \verb|env1| che in \verb|env2| applica l'operazione \verb|max| fra gli stati mentre per quelle presenti solo in \verb|env1|, le ritorna così come sono;
    \item \verb|Environment seq(Environment env1, Environment env2)| si comporta come \verb|max|, applicando però l'operatore \verb|seq| fra gli stati delle variabili;
    \item \verb|Environment par(Environment env1, Environment env2)| assume che \verb|env1| ed \verb|env2| siano ambienti con un solo scope (perché solo in tale contesto viene invocata questa funzione), per ogni variabile presente in entrambi gli ambienti viene fatto il \verb|par| fra gli stati, altrimenti le variabili vengono semplicemente aggiunte all'ambiente risultante.
    \item \verb|Environment update(Environment env1, Environment env2)| assume che \verb|env1| e \verb|env2| abbiano rispettivamente un livello di annidamento $\geq{} 1$ e $= 1$.
    Se \verb|env2| è vuoto oppure non ha variabili definite viene ritornato \verb|env1| (caso base) altrimenti, $\forall u \in$ \verb|env2| prima si rimuove $u$ da \verb|env2|, successivamente se $u \in top($\verb|env1|$)$ allora \verb|env1|$[u \rightmapsto{} u.type]$ e viene ritornato \verb|update(env1, env2)| (è una chiamata ricorsiva, con \verb|env1| e \verb|env2| che sono stati aggiornati), altrimenti effettua un serie di chiamate ricorsive riassumibili nella formula \verb|update(update(env,| $[u \rightmapsto{} u.type]$\verb|)|$\cdot{}\; top($\verb|env1|$)$\verb|, env2)| con \verb|env1| $\leftarrow$ \verb|env| $\cdot{}\; top($\verb|env1|$)$
    \item \verb|ArrayList<SemanticError> checkVariableStatus(IdNode variable,|\\ \verb|BiFunction<Effect, Effect, Effect> rule, Effect effectToApply)| che dato un nodo identificatore dell'AST, lo ricerca all'interno della \textit{Symbol Table}, e applica \verb|rule| (che può essere \verb|max|, \verb|seq| o \verb|par| definite in \hyperref[ss:effect-effetti]{Sezione 4.1.1 Classe \texttt{Effect}}) fra lo stato di \verb|id| e l'effetto \verb|effectToApply|, eventualmente ritornando un errore\footnote{La scelta di ritornare una lista di \texttt{SemanticError} è fatta per evitare di fare controlli di nullità sul risultato che appesantirebbe la lettura del codice e sarebbe prono ad errori.} in caso l'effetto risultante da \verb|rule| sia \verb|ERROR|.
\end{itemize}

\section{Controllo degli effetti}\label{s:controllo-effetti}
L'analisi degli effetti viene fatta all'interno del metodo \verb|checkSemantics| per avere accesso alla \textit{Symbol Table} che, essendo appena stata creata o aggiornata dal processo di analisi semantica, contiene tutto il necessario.
Svolgere questo processo al di fuori di \verb|checkSemantics| avrebbe richiesto la creazione di una copia dell'ambiente in ogni nodo dell'AST, in maniera poco efficiente.\\
\noindent
L'implementazione della parte di analisi degli effetti ricalca quanto visto durante le lezioni sull'analisi degli effetti del corso di ``Compilatori e interpreti", i cui riferimenti alle slide si trovano in fondo al presente documento.
In particolare, nelle prossime sezioni, vengono descritti più in dettaglio alcuni dei controlli più rilevanti.

\subsection{Uso di variabili non inizializzate}\label{s:uso-variabili-non-inizializzate}
Per impedire l'utilizzo di variabili non inizializzate è stato implementato un controllo nell'espressioni che accedono alle variabili.
Di seguito viene presentato la porzione di codice che se ne occupa:
\begin{lstlisting}[language = Java, escapeinside={(*@}{@*)}]
@Override
public ArrayList<SemanticError> checkSemantics(Environment env) {
    ArrayList<SemanticError> errors = new ArrayList<>();

    ...

    if (lhs.getId().getStatus(lhs.getDereferenceLevel()).equals(Effect.INITIALIZED)) {
        errors.add(new SemanticError(lhs + " is used prior to initialization."));
    }

    ...

    return errors;
}
\end{lstlisting}

\subsection{Dichiarazione e chiamata di funzione con metodo di punto fisso per il calcolo degli effetti sugli argomenti}\label{s:dichiarazione-chiamata-funzione-mpf}
Il linguaggio \textbf{SimpLanPlus}, come detto, supporta la ricorsione. Affinché gli effetti applicati agli argomenti della funzione siano calcolati correttamente, è stato implementato un \textbf{metodo di punto fisso (MPF)} per il loro calcolo.\\
\noindent
Il MPF è implementato nel contesto dell'analisi degli effetti sulla dichiarazione di funzione, ma il suo funzionamento è strettamente correlato a quello della chiamata di funzione, considerando la presenza della ricorsione.
Andando con ordine, illustriamo passo dopo passo il controllo sulla dichiarazione:
\begin{enumerate}
    \item viene creata una copia dell'ambiente (\verb|Environment|) prima dell'analisi del blocco di istruzioni correlate alla funzione;
    \item vengono impostati a $rw$ gli stati dei parametri formali (questo per permetterne il controllo ed evitare gli errori semantici presentati nella precedente sezione);
    \item viene analizzato il blocco di istruzioni e grazie a ciò gli argomenti della funzione vengono aggiornati con lo stato che ottengono dopo una singola chiamata di funzione;
    \item se gli effetti applicati agli argomenti sono diversi dagli effetti di partenza, si procede con il prossimo punto, altrimenti si passa al punto 7.;
    \item viene ripristinato l'ambiente che era stato copiato al punto 1. inserendo nelle \textit{entry} relative agli argomenti i nuovi effetti calcolati;
    \item aggiornato l'ambiente, si analizza nuovamente il blocco con i nuovi effetti degli argomenti calcolati all'iterazione precedente e successivamente si passa al punto 4.;
    \item viene effettuato \verb|popScope| e successivamente aggiornati gli effetti degli argomenti della funzione presenti nello scope in cui è stata originariamente definita.
\end{enumerate}
Per quanto riguarda la chiamata di funzione:
\begin{enumerate}
    \item vengono recuperate tutte le espressioni (incluse anche variabili) non puntatori e ci si assicura che il corrispettivo parametro formale non sia in stato $\top$;
    \item viene creata una copia dell'ambiente, chiamata \verb|e1|, in cui viene aggiornato l'effetto a $rw$ di tutte le variabili facenti parte delle precedenti trovate espressioni;
    \item viene creata un'ulteriore copia dell'ambiente, \verb|e2|;
    \item per ogni puntatore passato come argomento alla funzione viene creato un nuovo ambiente temporaneo in cui salvare solo una \textit{entry}, relativa a quel puntatore, il cui effetto è il \verb|par| fra l'effetto trovato nell'ambiente di partenza e quello trovato nel corrispondente parametro attuale della funzione da chiamare;
    \item tutti questi ambienti temporanei generati al punto precedente vengono messi in \verb|par| fra loro (a cascata) e il risultato viene memorizzato nel precedentemente creato ambiente \verb|e2| (nel caso non ci fossero ambienti temporanei creati, \verb|e2| $= \emptyset{}$);
    \item infine, viene invocato \verb|update| con parametri \verb|e1| e \verb|e2| e il risultato è il nuovo ambiente per le successive istruzioni.
\end{enumerate}

\end{document}

